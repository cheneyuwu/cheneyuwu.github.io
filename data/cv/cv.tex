\documentclass{cv}

\usepackage[left=0.3in,top=0.3in,right=0.3in,bottom=0.3in]{geometry} % Document margins
\usepackage{enumitem}

\begin{document}

\printname{Yuchen Wu}
\printaddress{Toronto, Ontario, Canada}
\printaddress{+1-519-694-7627 \\ \href{mailto:cheneyuwu@gmail.com}{cheneyuwu@gmail.com}}
\printaddress{\href{https://cheneyuwu.github.io/}{https://cheneyuwu.github.io/}}

\begin{rSection}{ACADEMIC HISTORY}
  \item {\bf MASc in Aerospace Science and Engineering} \hfill {Sept. 2020 - Present}\\
  University of Toronto Institute for Aerospace Studies (UTIAS), Canada\\
  Supervisor: Prof. Timothy D. Barfoot\\
  Thesis: \textit{VT\&R3: Generalizing the Visual Teach \& Repeat Navigation Framework}

  \item {\bf BASc in Engineering Science (Robotics)} \hfill {Sept. 2015 - Apr. 2020}\\
  University of Toronto, Canada\\
  CGPA: 3.93 / 4.0 (High Honours)\\
  Supervisor: Prof. Florian Shkurti and Prof. Jonathan Kelly\\
  Thesis: \textit{Combining Reinforcement Learning and Imitation Learning through Reward Shaping for Continuous Control}
\end{rSection}

\begin{rSection}{EMPLOYMENT HISTORY}
  \item \textbf{Intel Corporation}, Toronto, Canada  \hfill {May 2018 - May 2019}\\
  Software Engineer Intern\\
  Product: \textit{Intel HLS Compiler} and \textit{Intel FPGA SDK for OpenCL}
  \vspace{-0.5em}
  \begin{itemize}[noitemsep,topsep=0pt]
    \item Intel HLS Compiler: a high-level synthesis (HLS) tool that takes in untimed C++ code and generates production-quality register transfer level (RTL) code optimized for Intel FPGAs.
    \item Intel FPGA SDK for OpenCL: development environment that enables software developers to accelerate applications by targeting heterogeneous platforms with Intel CPUs and FPGAs.
  \end{itemize}
\end{rSection}

\begin{rSection}{SKILLS}
  \item
  \begin{tabular}{ @{} >{\bfseries}l @{\hspace{6ex}} l }
    Communication      & Chinese (Mandarin), English                                                         \\
    Programming        & C/C++, Python, JavaScript, Java                                           \\
    Software/Libraries & MATLAB, Robot Operating System (ROS), MuJoCo, OpenCV, PyTorch, TensorFlow \\
  \end{tabular}
\end{rSection}

\begin{rSection}{PUBLICATIONS}
  \item \textbf{Along Similar Lines: Local Obstacle Avoidance for Long-term Autonomous Path Following}\\
  Jordy Sehn, \textbf{Yuchen Wu}, Timothy D. Barfoot\\
  Submitted to \textit{IEEE International Conference on Robotics and Automation (ICRA)}, 2023

  \item \textbf{Picking Up Speed: Continuous-Time Lidar-Only Odometry using Doppler Velocity Measurements}\\
  \textbf{Yuchen Wu}, David J. Yoon, Keenan Burnett, Soeren Kammel, Yi Chen, Heethesh Vhavle, Timothy D. Barfoot\\
  Submitted to \textit{IEEE Robotics and Automation Letters (RA-L)}, 2022

  \item \textbf{Are We Ready for Radar to Replace Lidar in All-Weather Mapping and Localization?}\\
  Keenan Burnett*, \textbf{Yuchen Wu}*, David J. Yoon, Angela P. Schoellig, Timothy D. Barfoot\\
  \textit{IEEE Robotics and Automation Letters (RA-L)}, 2022
  % \textit{IEEE/RSJ International Conference on Intelligent Robots and Systems (IROS)}, 2022

  \item \textbf{Boreas: A Multi-Season Autonomous Driving Dataset}\\
  Keenan Burnett, David J. Yoon, \textbf{Yuchen Wu}, Andrew Zou Li, Haowei Zhang, Shichen Lu, Jingxing Qian, Wei-Kang Tseng, Andrew Lambert, Keith Y.K. Leung, Angela P. Schoellig, Timothy D. Barfoot\\
  Accepted by \textit{International Journal of Robotics Research (IJRR)}

  \item \textbf{Shaping Rewards for Reinforcement Learning with Imperfect Demonstrations using Generative Models}\\
  \textbf{Yuchen Wu}, Melissa Mozifian, Florian Shkurti\\
  \textit{IEEE International Conference on Robotics and Automation (ICRA)}, 2021
\end{rSection}

\begin{rSection}{OPEN-SOURCE PROJECTS}
  \item \textbf{Visual Teach and Repeat 3 (VT\&R3)} \hfill \href{https://github.com/utiasASRL/vtr3}{https://github.com/utiasASRL/vtr3}
  \vspace{-0.25em}
  \begin{itemize}[noitemsep,topsep=0pt]
    \item An end-to-end navigation system for long-range and long-term mobile robot path following using a lidar, radar, or camera as the primary sensor.
  \end{itemize}
\end{rSection}

\begin{rSection}{AWARDS}
  \item \textbf{Vector Scholarship in AI}, Vector Institute \hfill 2020
  \item \textbf{CRA Outstanding Undergraduate Researchers Honorable Mentions} \hfill 2020
  \item \textbf{University of Toronto Dean's Honours List} \hfill 2015 - 2020
  \item \textbf{University of Toronto Excellence Awards (UTEA)} \hfill 2019
  \item \textbf{Garnet W. Mckee - Lachlan Gilchrist Scholarship}, UofT \hfill 2017
\end{rSection}

\begin{rSection}{STUDENT ACTIVITIES}
  \item \textbf{UofT aUToronto Team}, Student Advisor,  \hfill Sept. 2021 - Jun. 2022
  \vspace{-0.25em}
  \begin{itemize}[noitemsep,topsep=0pt]
    \item 1st place overall in the first competition of the four-year SAE AutoDrive Challenge Series II.
  \end{itemize}
  \item \textbf{ROB310 Mathematics for Robotics}, Teaching Assistant \hfill Fall 2021
  \item \textbf{University of Toronto}, Research Assistant \hfill May 2019 - Sept. 2019
  \vspace{-0.25em}
  \begin{itemize}[noitemsep,topsep=0pt]
    \item Supervisor: Prof. Florian Shkurti at the Department of Computer Science
    \item Worked on reinforcement and imitation learning for control.
  \end{itemize}
  \item \textbf{UofT Machine Intelligence Student Team}, Academic Lead \hfill Sept. 2018 - May 2019
  \vspace{-0.25em}
  \begin{itemize}[noitemsep,topsep=0pt]
    \item Built a machine learning community for undergrad students.
    \item Organized MIST101, a workshop on machine learning fundamentals.
  \end{itemize}
  \item \textbf{University of Toronto}, Research Assistant \hfill May 2017 - Sept. 2017
  \vspace{-0.25em}
  \begin{itemize}[noitemsep,topsep=0pt]
    \item Supervisor: Prof. Jianwen Zhu at the Department of Electrical and Computer Engineering
    \item Worked on accelerating the training and inference of deep CNN on multi-core CPU.
  \end{itemize}
  \item \textbf{National University of Singapore}, Research Assistant \hfill May 2016 - July 2016
  \vspace{-0.25em}
  \begin{itemize}[noitemsep,topsep=0pt]
    \item Supervisor: Prof. Shailendra Joshi at the Department of Mechanical Engineering
    \item Worked on computational modeling and analysis of nano/micro lattice structure.
  \end{itemize}
\end{rSection}

\end{document}
